\chapter{Introduction to Stream Processing}

\section{What is Stream Processing?}
- Definition: A stream is a sequence of unbounded tuples of the form (a1, a2,
... an, t) generated continously in time. ai=attribut, t=time

- Stream processing = processing data before storing. Batch systems like hadoop
provide processing after storingS- Sream processing = processing data before
storing. HBatch sstems like hadoop provide processing after storing  

- a stream processing system is built out of multiple units called a processing
elemen (PE). Each PE receive input from their input queues, does some
computation on the input using its local state and produce output to their
output queues. PE communicate always through messaging with other PEs.

- Most important attributes which however are competing each other : (low)
Latency of the system and high availability. High availability by recovering
from failures is critical for a Stream processing system - recovery should be
fast and efficient. Data should partitioned and handled in parallel for large
volumes of data. The partitioning strategy of a system  affects how the system
handles the data in paralell and how the system can scale. 

\subsection{Stream sources}
-From traditional PCs and Smartphones to a lot of sensors who are connected to
the interne -> Internet of Things!

\section{Real-time versus Batch Processing}

As the trend shows, the needs of a big data environment can't be fulfilled with 
traditional batch processing anymore. Instead, real-time processing becomes more 
important than ever to achieve results from queries in minutes, even seconds. 
\cite{bange2013big}
\\ \\
Traditional batch processing systems nowadays are distinguished between
map-reduce based and non map-reduce based systems and typically consists of two
stages, data integration and data analytics. The process of data
extraction-transformation-load (ETL  [glossar]), faced in the data integration stage runs
at a regular time interval, such as daily, weekly or monthly. The process, which
analyzes the data residing in a data store is being face in the data analytics
stage and becomes challenging when data size grows and systems may not be able
to process results within a time limit.\cite{Liu:2014:SRP:2628194.2628251}
\\ \\
In real-time processing fashion, systems will address the data integration stage
with continual input of data. Processing in near-real-time [glossar] to present 
results within seconds is being addressed in the data analytics stage. Thus,
real-time processing gives organization the ability to take immediate action
for those times when acting within seconds or minutes is significant.
\cite{PrpSvyOfDSPS}
The lambda architecture introduces a new
paradigm for big data whit which those needs can be achieved. Any query is
answered through the serving layer by querying both the speed and the batch
layer. Where the badge layer computes views on the current collected data and
is being outdated at the end of it's computation, the speed layer closes this 
gap by constantly processing the most recent data in near real-time fashion. 
\cite{marz2015big}
\\
%todo: point to the need of messaging systems that deliver data continously


\section{(Distributed) Stream processing systems}

\section{Complex event processing (CEP)}
In literatur there is often a confusion about the difference between the terms
complex event processing and stream event processing. Both systems work on
events and produce results based on the properties of the events... 
- Zitat: Combines data from multiple sources  to detect patterns and attempt to
identify either opportunities or threats. The goal is to identify significant
events and respond fast. Sales leads, orders or customer service calls are
examples.\\


\section{Batch processing}
- Processing data after storing 
-Zitat: Processing hight volumes of data is where a group of transaction is collected
over a period of time. Data is collected, entered, processed and the batch
result are produces. -> Real-time virtually impossible
- Apache Hadoop\\
- Data to be stored and analyzed by the batch processing system to get a deeper
understanding and discovering patterns
\subsection{Map Reduce model }

\section{Lambda Architecture}
\cite{PrpSvyOfDSPS}

