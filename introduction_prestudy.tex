\chapter{Motivation}
Regarding distributed systems and the involved technologies, a lot of different
terms and definitions have been given in the literature. First part of this
preparatory study aims to achieve more clarity in this confusing \textit{jungle of
terminology}, whereas the focus lies on topics that are related to Apache
Kafka including Messaging, Logging and Stream Processing. After defining this
basic concepts for referencing during this thesis, we go deeper into
implementations of specific systems and compare them to each other. In fact, the
goal of this survey is to show the differences of the most related message
broker alternatives in contrast to Apache Kafka. Finally we will give a more
detailed insight in the technology and components of Kafka itself.\\

\todo[inline]{illustration of the terminology jungle, with all the words we have
met during prestudy and which we want to get in order}


In advance, we will start with the definition of the term distributed systems. A
distributed system consists of a collection of multiple autonomous components,
connected through network, which enables to coordinate  (e.g. computers) their
activities and to share resources. To a user, these systems are perceived as a
single, integrated computing facility. Other than centralised systems,
distributed systems has it's advantages in economics, performance and
scalability and reliability. In fact, this approach allows us to build reliable
applications, scaled using low cost computers and thus can serve massive
performance needs whether on demand or continuously.\cite{POSA1}\cite{TAN06}
