\chapter{Introduction to Message Broker}
\section{Distributed Systems}
A distributed system consists of a collection of multiple autonomous components,
connected through network, which enables to coordinate their activities and to
share the resources. To a user, these systems are perceived as a single,
integrated computing facility \cite{TAN06}. Other than centralised systems,
distributed systems has it's advantages in economics, performance and
scalability, inherent distribution and reliability \cite{POSA1}.

\subsection{Messaging}
Introducing a indermediate component between to distributed clients, messaging
reduces the close coupling of other communication styles based on Sockets or
RMI. This additional component can be named as Message-Oriented Middleware (MOM)
and is all about a queueing system where applications can inserting messages
which are forwarded and delivered to another program. Thereby the main
characteristic of a MOM is that its supports a storage capacity which leads to a
way of communication where it is not required that the collaborating endpoints
are active during the transmission of a message.\cite{TAN06}
 
\subsubsection{Channels and Topics}

\subsubsection{Publish-Subscribe}

\subsection{Other communication styles}
\subsubsection{Remote Procedure Call}
\subsubsection{Stream-Oriented}

\subsection{Definition of Real-Time}

\section{Broker Pattern}
Different patterns were introduced related to distributed systems. The most
relevant to build a fully-fledged message broker system proves to be the
\textit{Broker pattern} which can be used to structure distributed systems with
decoupled components \cite{POSA1}. 

\section{What is a Message Broker?}
We saw that a \textit{Broker} acts like a mediator between to collaborating
applications which do not need to know each other. The same does a message
broker, of course in a message-oriented distributed system whereas the broker
handles incoming messages from a source to a target and backwards. 

In such an environment it is essential that existing and new applications can be
integrated into a single, coharent system at runtime. Often these applications
are not speaking the same language and it need kind of a gateway for
transforming messages into a format that can be unterstood by the receiver.
\cite{TAN06}

Note: Gem [TAN06], ist ein Message Broker lediglich ein Konvertierer für
Messages in unterschiedlichen Formate und nicht essentiell für ein einfaches
Queueing system. Wenn man jedoch das Broker Pattern[POSA1] als Grundlage nimmt
ist der Broker die zentrale Komponente die auch das Publish / Subscribe
Verhalten kontrolliert. 

TODO: verschiedene Definitionen aufnehmen vergleichen, Schluss ziehen
\section{Providers}
\subsection{JMS}
\subsection{AMQP}
\subsection{Kafka}
\\
