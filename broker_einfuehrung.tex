\chapter{Introduction to Message Broker} 
\section{Distributed Systems} 
A distributed system consists of a collection of multiple autonomous components,
connected through network, which enables to coordinate their activities and to
share the resources. To a user, these systems are perceived as a single,
integrated computing facility \cite{TAN06}. Other than centralised systems,
distributed systems has it's advantages in economics, performance and
scalability, inherent distribution and reliability \cite{POSA1}.
In fact, this approach allows us to build reliable applications, 
scaled using low cost computers and thus can serve massive performance needs 
whether on demand or continuously.

\subsection{Messaging} 
Introducing an intermediate component between distributed
clients, messaging reduces the close coupling of other communication styles just
based on Sockets or RMI. This additional component can be named as
Message-Oriented Middleware (MOM) and is all about passing messages in any
format from one application to another whereas the parties do not need to know
each other directly. Thereby the main characteristic of a MOM is that its
supports a storage capacity which leads to a way of communication where it is
not required that the collaborating endpoints are active during the entire
transmission of a message.\cite{TAN06}
 
\subsubsection{Channels and Topics} 
Basically Message-Oriented Middleware is based on queues where applications insert
messages which are forwarded and delivered to another destination. 
\\ 
Topic unterscheided zwischen Publisher und Subscriber
\\ Channel unterscheided zwischen Producer and Consumer. Wurde eine Nachricht
konsumiert, wird sie aus der Queue gelöscht. 

\subsection{Open Standards and Protocols} 
\begin{itemize} 
	\item Advanced Message Queuing Protocol (AMQP) 
	\item JMS	
	\item XML und SOAP
\end{itemize}

\subsection{Other communication styles} \subsubsection{Remote Procedure Call}

\subsubsection{Stream-Oriented}
The type of communication where timing plays a crucial role is often referred as a
stream oriented communication. 


\subsection{Definition of Real-Time}

\section{What is a Message Broker?}

\subsection{Broker Pattern} 
Different patterns were introduced related to distributed systems. The most
relevant to build a fully-fledged message broker system proves to be the
\textit{Broker pattern} which can be used to structure distributed systems with
decoupled components \cite{POSA1}. 

\subsection{Definition Message Broker}
We saw that a \textit{Broker} acts like a mediator between to collaborating
applications which do not need to know each other. The same does a message
broker, of course in a message-oriented distributed system whereas the broker
handles incoming messages from a source to a target and backwards. 

In such an environment it is essential that existing and new applications can be
integrated into a single, coharent system at runtime. Often these applications
are not speaking the same language and it need kind of a gateway for
transforming messages into a format that can be unterstood by the receiver.
\cite{TAN06}

Note: Gem [TAN06], ist ein Message Broker lediglich ein Konvertierer für
Messages in unterschiedlichen Formate und nicht essentiell für ein einfaches
Queueing system. Wenn man jedoch das Broker Pattern[POSA1] als Grundlage nimmt
ist der Broker die zentrale Komponente die auch das Publish / Subscribe
Verhalten kontrolliert. 

TODO: verschiedene Definitionen aufnehmen vergleichen, Schluss ziehen \\

\section{Implementations}
\begin{itemize}
	\item ActiveMQ 
	\item RabbitMQ
	\item ZeroMQ
	\item Apache Kafka
	\item ... 
\end{itemize}
\section{Survey}


