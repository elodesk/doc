\chapter{Conclusion}
\section{Results}
\subsection{Prestudy}
The first result of this thesis is the prestudy documentation, which
resumes gathered knowledge in the familiarization phase of this work.
It gives an insight in messaging fundamentals and  takes a closer look to Apache
Kafka and related topics.

\subsection{Protocol Implementation}
A fundamental result of this thesis is the implementation of the Apache Kafka
protocol in Haskell. The design decision of separating protocol related code
from the broker implementation leads to a isolated product which can be used as
library for different project. The code is provided through an open sourced
repository for further development. \todo{ref to Github}

A complete implementation of the Apache Kafka Protocol would go beyond the scope
of this thesis as the focus lies not only in implementing the protocol but also
to provide broker functionality (see Chapter \ref{chap:broker}). Thus, most
important is the ability to produce and fetch messages. The following list gives
an overview of what part of the protocol is implemented and what remains open:

\begin{itemize}
    \item Metadata API
    \begin{itemize}
        \tick Topic Metadata Request
        \tick Metadata Response
    \end{itemize}
    \item Produce API
    \begin{itemize}
        \tick Produce Request
        \tick Produce Response
    \end{itemize}
    \item Fetch API
    \begin{itemize}
        \tick Fetch Request
        \tick Fetch Response
    \end{itemize}
    \item Offset API
    \begin{itemize}
        \fail Offset Request
        \fail Offset Response
    \end{itemize}

    \item Offset Commit/Fetch API
    \begin{itemize}
        \fail Consumer Metadata Request
        \fail Consumer Metadata Response
        \fail Offset Commit Request
        \fail Offset Commit Response
        \fail Offset Fetch Request
        \fail Offset Fetch Response
    \end{itemize}
\end{itemize}

\section{Haskell Message Broker}
The main approach of this work is the implementation of a message broker in
Haskell. The resulting application provides a server with basic functionality
in networking and persisting messages. The broker supports Kafka clients as
it is based on the protocol implementation mentioned above.

..?: 

Producing Messages: 
\begin{itemize}
        \tick Publishing messages to specific topic and partition
        \tick Persist messages in log based file structure
        \tick Support batched messages 
        \tick Support producing for multiple topics and partitions
        \fail Configurable Acknowledgements and Timeout
\end{itemize}

Fetch Messages: 
\begin{itemize}
        \tick Consuming messages of a specific topic and partition
        \tick Request Messages depending on given offset
        \fail Support consuming from multiple topics and partitions 
        \fail Support configurable min and max bytes for fetched data
        \fail Support maximum amount of time to block, waiting if insufficient
        data is available
\end{itemize}
Clustering: The scope lies on a single broker system with a producer and
consumer API. There is no support for broker replication yet. 


\section{Outlook}
-> What are the next step? 

- Further optimizations 
- Implementation and handling of remaining API's 
- Replication and Integration With Apache Zookeeper 
- Ausbauen von Client API 
-> Open Source 

-> 


