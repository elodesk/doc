\chapter{Projectmanagement}

\section{Introduction}

\subsubsection*{Zweck dieses Dokuments}

\subsubsection*{Zweck und Ziel dieser Arbeit}

\subsubsection*{Lieferumfang}

\subsubsection*{Annahmen und Einschränkungen}

\section{Projektorganisation}

\subsubsection*{Struktur}

\begin{tabular}[t]{|l|l|l|} \hline
\textbf{Name} & \textbf{E-Mail} & \textbf{Aufgabe und Verantwortungen} \\ \hline \hline
\end{tabular}

\subsubsection*{Externe Schnittstellen}

\begin{tabular}[t]{|l|l|l|} \hline
\textbf{Name} & \textbf{E-Mail} & \textbf{Aufgabe und Verantwortungen} \\ \hline \hline
\end{tabular}

\subsubsection*{Sitzungen}


\begin{itemize}
\item 
\end{itemize}

\section{Managementabläufe}

\subsection{Zeiterfassung}

%Die Zeiterfassung wird von den Projektmitarbeitern in Redmine erfasst.
%Redmine ist über folgenden Link erreichbar:
%\url{http://152.96.56.42/redmine/}. Die Projektmitarbeiter haben einen
%persönlichen Zugang  um die Daten zu erfassen. Der Betreuer hat
%ebenfalls ein Zugang, mit welchem er die Fortschritte mitverfolgen kann.
%Auf Redmine ist der aktuelle Stand des Projekts zu sehen.

\subsection{Arbeitspakete und zeitliche Planung}

% Wir haben die geplante Arbeit in Arbeitspakete aufgeteilt. Diese
% Arbeitspakete wurden in Redmine als Tickets erfasst und sind jeweils dem
% Meilenstein zugeordnet in dem sie fertiggestellt werden sollten. In der
% Abbildung \ref{fig:ArbeitspaketeGeplant} ist eine Übersicht über die
% Arbeitspakete und die geplante Dauer der Arbeitspakete zu sehen.
% In der Abbildung \ref{fig:ArbeitspaketeGeplantGANTT} sind die Pakete im GANTT Format
% dargestellt.
% 
% \begin{landscape}
% \begin{footnotesize}
% 
% \begin{figure}[h]
%   \centering
%   \includegraphics[width=24cm]{images/zeitplanung.png}
%   \caption{Uebersicht über die geplanten Arbeitspakete}
%   \label{fig:ArbeitspaketeGeplant}
% \end{figure}
% 
% \begin{figure}[h]
%   \centering
%   \includegraphics[width=24cm]{images/gantt.png}
%   \caption{Uebersicht über die geplanten Arbeitspakete im GANTT Format}
%   \label{fig:ArbeitspaketeGeplantGANTT}
% \end{figure}
% 
% \end{footnotesize}
% \end{landscape}
% \newpage

\subsection{Meilensteine}

Wichtige Daten im Projekt wurden mit den Meilensteinen in Tabelle
\ref{tab:MeilensteineZiele} festgesetzt. Nach jedem geplanten Ende eines
Meilensteines, wird ein Soll-Ist Vergleich durchgeführt und im Kapitel
Projektstandverfolgung festgehalten.

\begin{tabular}[t]{|l|l|p{8cm}|}\hline
\textbf{Meilenstein} & \textbf{Datum} & \textbf{Ziele} \\ \hline \hline
Projektstart & 16.02.2015 & \\ \hline
M1 Vorstudie Message broker abgeschlossen & 22.03.2015 &
  \begin{itemize}
    \item \ldots
  \end{itemize} \\ \hline
\end{tabular}
\captionof{table}{Meilensteine und deren Ziele}
\label{tab:MeilensteineZiele}


\newpage
\section{Risikomanagement(Plagiat)}

In der Tabelle \ref{tab:Risiken} sind die Risiken ersichtlich, welche
unser Projekt beeinflussen können.

\begin{tabular}[t]{|p{3cm}|p{3cm}|r|r|r|p{3cm}|p{3cm}|}\hline
\textbf{Risiko} &
  \textbf{Auswirkung} &
  \begin{sideways} \textbf{Wahrscheinlichkeit } \end{sideways} &
  \begin{sideways}\textbf{Schaden} \end{sideways} &
  \begin{sideways}\textbf{Risiko} \end{sideways} &
  \textbf{Vorbeugung} & \textbf{Konsequenzen} \\ \hline \hline
Datenverlust &
  verlorene Arbeit &
  0.1 & 0.9 & 0.1 &
  regelmässige Backups &
  Arbeit in Sonderschicht nachholen \\ \hline
Ausfall eines Projektmitarbeiters &
  Nichteinhaltung des Terminplans &
  0.1 & 0.9 & 0.1 &
  Nicht vermeidbar &
  Mehrarbeit für nicht ausgefallenen Mitarbeiter \\ \hline
Kommunikations"-probleme &
  Zeitverlust, zielloses Arbeiten &
  0.1 & 0.3 & 0.0 &
  Teambildungs"-massnahmen &
  Diskussion suchen, Betreuer informieren \\ \hline
\end{tabular}
\captionof{table}{Risiken}
\label{tab:Risiken}

Sollte trotz den vorbeugenden Massnahmen ein zeitlicher Schaden
entstehen, muss die Projektplanung unter Umständen angepasst werden.

\section{Qualitätsmanagement}

%Achtungb Plagiat: 
%Um die Arbeitsergebnisse qualitativ auf einem hohen Niveau zu halten,
%arbeiten die Projektmitglieder nach dem Vier-Augen-Prinzip. Ein Dokument
%wird immer von beiden durchgelesen. Allfällige Änderungen werden gleich
%bilateral diskutiert und allenfalls angebracht. Somit soll erreicht
%werden, dass beide Projektmitglieder mit den Ergebnissen einverstanden
%und zufrieden sind.  Um Programmieraufgaben durchzuführen, kann
%teilweise der Ansatz von Pairprogramming eingesetzt werden. Dies führt
%dazu, dass sich beide mit dem Code auskennen.

\section{Projektstandverfolgung}

\subsection{Meilenstein 1}

\section{Zeitauswertung}

\subsection{Projektstunden pro Woche}

\subsection{Projektstunden aufsummiert}

\subsection{Projektstunden pro Projektmitglied}

\subsection{Stunden pro Tätigkeitsbereich}
