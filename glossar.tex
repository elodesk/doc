\newglossaryentry{Channel}
{
    name=Channel,
    description={
        Messaging applications transmit data through a Message Channel, a virtual
        pipe that connects a sender to a receiver. A newly installed messaging
        system does not contain any channels; you must determine how your
        applications need to communicate and then create the channels to facilitate
        it.
    }
}
\newglossaryentry{Msg}
{
    name=Message,
    description={
        A Message is an atomic packet of data that can be transmitted on a
        channel. Thus to transmit data, an application must break the data into
        one or more packets, wrap each packet as a message, and then send the
        message on a channel. Likewise, a receiver application receives a
        message and must extract the data from the message to process it. The
        message system will try repeatedly to deliver the message (e.g.,
        transmit it from the sender to the receiver) until it succeeds.
    }
}
\newglossaryentry{Socket}
{
    name=Socket,
    description={
        Communication endpoint to which an application can write data
        that are to be sent out over the underlying network, and from which incoming end
        data can be read. \cite{TAN06}
    }
}
\newglossaryentry{Queue}
{
    name=Queue,
    description={-}
}


\newglossaryentry{FIFO}
{
    name=FIFO,
    description={-}
}

\newglossaryentry{ETL}
{
    name=ETL,
    description={Extract (data from homogeneous or heterogeneous data sources), 
         Transform (the data for storing it in proper format or structure
         for querying and analysis purpose)
         Load (it into the final target)
    }
}

\newglossaryentry{file system}
{
    name="file system",
    description={
        A file system is used to control how data is stored and retrieved. Some
        file systems are used on local data storeage devices; others privde file
        access via a newtwork protocol.
    }
}

\newglossaryentry{page cache}
{
    name="page cache",
    description={
        Page cache (often called disk cache) is a cache of disk-backed pages
        kept in main memory (RAM) by the operating system for quicker access.
    }
}

\newglossaryentry{replication}
{
    name="replication",
    description={
        Replication in computing involves sharing information so as to ensure
        consistency between redundant resources, such as software or hardware
        components, to improve reliability, fault-tolerance, or accessibility
     }
}
