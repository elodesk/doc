\chapter{Evaluation}
\section{Protocol Implementation}
A complete implementation of the Apache Kafka Protocol would
go beyond the scope of this thesis as the focus lies not
only in implementing the protocol but also to provide broker
functionality (see Chapter \ref{chap:broker}). Thus, most important is
the ability to produce and fetch messages. The following list gives 
an overview of what part of the protocol is implemented and what remains open:

\begin{itemize}
    \item Metadata API
    \begin{itemize}
        \tick Topic Metadata Request
        \tick Metadata Response
    \end{itemize}
    \item Produce API
    \begin{itemize}
        \tick Produce Request
        \tick Produce Response
    \end{itemize}
    \item Fetch API
    \begin{itemize}
        \tick Fetch Request
        \tick Fetch Response
    \end{itemize}
    \item Offset API
    \begin{itemize}
        \fail Offset Request
        \fail Offset Response
    \end{itemize}

    \item Offset Commit/Fetch API
    \begin{itemize}
        \fail Consumer Metadata Request
        \fail Consumer Metadata Response
        \fail Offset Commit Request
        \fail Offset Commit Response
        \fail Offset Fetch Request
        \fail Offset Fetch Response
    \end{itemize}
\end{itemize}

The implemented APIs are fully compatible with original Apache Kafka.
Therefore Haskell clients which using the client library of the protocol
implementation (see section \ref{sec:impl-prot-client}), can either communicate
with the HMB or Apache Kafka broker (depending on the set up IP and Port
configuration). 

\newpage
\section{Broker Features}
The scope of this thesis lies on a single broker system with a producer and
consumer API. Therefore it not supports broker replication nor a integration
with Apache Zookeeper like Apache Kafka yet.

\begin{itemize}
    \item Network Layer 
    \begin{itemize}
        \tick Establishing multiple client connections
        \tick Receiving Request over socket
        \tick Sending Responses over socket
    \end{itemize}
    \item API Layer
    \begin{itemize}
        \tick Parsing incoming requests
        \tick Catching and handling errors at runtime 
        \tick Handling Produce requests
        \tick Handling Fetch requests
        \tick Handling Metadata requests
        \fail Handling Offset requests
        \fail Handling Offset Commit/Fetch requests
    \end{itemize}
    \item Producing Messages
    \begin{itemize}
        \tick Publishing messages to specific topic and partition and persist
        them for further consumption
        \tick Support nested sequences of topics, partitions and messages per
        request
        \fail Support configurable Acknowledgements and Timeout
    \end{itemize}
    \item Fetching Messages 
    \begin{itemize}
        \tick Consuming messages of a topic and partition from a specific offset
        on
        \item Support nested sequences of topics and partitions
        \fail Support configurable min and max bytes for fetched data
        \fail Support maximum amount of time to block, waiting if insufficient
        data is available
    \end{itemize}
    \item Accessing Metadata of Topic
    \begin{itemize}
        \tick Requesting metadata for specific topic (or all)
        \tick Responding information about names of available topics
        \tick Responding information about broker instance
        \fail Responding topic or partition specific metadat
        \fail Responding informations about replication 
    \end{itemize}
    \fail Offset Management via Request
    \item Log Persistency
    \begin{itemize}
        \tick Persisting messages in log file
        \tick Guarantee ordering of messages through continuously offsets
        \tick Keep book of a log file with separate index file 
        \fail Log Compaction 
    \end{itemize}
    \item Optimizations 
    \begin{itemize}
        \tick Network throughput 
        \tick Caching
    \end{itemize}
    \fail Message Compression 
    \fail Consumer Groups 
    \fail Broker Recovery
    \fail Broker Replication and Zookeeper Integration
   \end{itemize}

\section{Broker Performance}
-> Test System

Broker Server Setup
\begin{verbatim}
Intel(R) Xeon(R) CPU E3-1245 v3 @ 3.40GHz
One 7200 RPM SATA drive
16GB of RAM
1Gb Ethernet 
\end{verbatim}

\subsection{Network Throughput}
The socket based communication must not be a bottleneck in a high performance
broker. A producer should be able to send with near the maximal throughput of
its local network card, whereas the broker needs to handle the incoming data
streams as fast as possible. If the limits of a single network connection is
reached, the next step to improve performance is to balance the data
stream on multiple replicated brokers.

To test the network throughput for this thesis, a benchmark producer
which identify the limits is provided.

\begin{lstlisting}
import qualified System.Entropy as E

main = do 
    -- Socket setup 
    -- ....

    randBytes <- E.getEntropy 1000
    let
\end{lstlisting}


The variance seems to be due to Linux's I/O management facilities that batch
data and then flush it periodically.
-> Durchgeführte Tests 

-> Resultat Network Throuput 

-> Resultat Writing to the log 
-> Vergleich Kafka (ref to Blog) 


\begin{table}[h]
\begin{tabular}{|c|c|c|c|c|c|c|l|}
\hline
{\bf Request Size} & {\bf Message Size} & {\bf Batch Size} & {\bf Encode} & {\bf Receive} & {\bf Decode} & {\bf WriteToDisk} & \multicolumn{1}{c|}{{\bf \begin{tabular}[c]{@{}c@{}}Throughput\\ (MB/s)\end{tabular}}} \\ \hline
                & 100                & 1                & x            & x
& x            & -                 & 25                                                                                       \\ \hline
                   & 100                & 10               & x            & x             & x            & -                 &                                                                                        \\ \hline
                   & 100                & 100              & x            & x             & x            & -                 &                                                                                        \\ \hline
                   & 100                & 1000             & x            & x             & x            & -                 &                                                                                        \\ \hline
                   & 100                & 10000            & x            & x             & x            & -                 &                                                                                        \\ \hline
                   & 1000               & 1                & x            & x             & x            & -                 &                                                                                        \\ \hline
                   & 1000               & 10               & x            & x             & x            & -                 &                                                                                        \\ \hline
                   & 1000               & 100              & x            & x             & x            & -                 &                                                                                        \\ \hline
                   & 1000               & 1000             & x            & x             & x            & -                 &                                                                                        \\ \hline
                   & 10000              & 1                & x            & x             & x            & -                 &                                                                                        \\ \hline
                   & 10000              & 10               & x            & x             & x            & -                 &                                                                                        \\ \hline
                   & 10000              & 100              & x            & x             & x            & -                 &                                                                                        \\ \hline
\end{tabular}
\end{table}

