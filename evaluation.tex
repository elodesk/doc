\chapter{Evaluation}

%\section{Broker Features}

%\begin{itemize}
%    \item Network Layer 
%    \begin{itemize}
%        \tick Establishing multiple client connections
%        \tick Receiving Request over socket
%        \tick Sending Responses over socket
%    \end{itemize}
%    \item API Layer
%    \begin{itemize}
%        \tick Parsing incoming requests
%        \tick Catching and handling errors at runtime 
%        \tick Handling Produce requests
%        \tick Handling Fetch requests
%        \tick Handling Metadata requests
%        \fail Handling Offset requests
%        \fail Handling Offset Commit/Fetch requests
%    \end{itemize}
%    \item Producing Messages
%    \begin{itemize}
%        \tick Publishing messages to specific topic and partition and persist
%        them for further consumption
%        \tick Support nested sequences of topics, partitions and messages per
%        request
%        \fail Support configurable Acknowledgements and Timeout
%    \end{itemize}
%    \item Fetching Messages 
%    \begin{itemize}
%        \tick Consuming messages of a topic and partition from a specific offset
%        on
%        \item Support nested sequences of topics and partitions
%        \fail Support configurable min and max bytes for fetched data
%        \fail Support maximum amount of time to block, waiting if insufficient
%        data is available
%    \end{itemize}
%    \item Accessing Metadata of Topic
%    \begin{itemize}
%        \tick Requesting metadata for specific topic (or all)
%        \tick Responding information about names of available topics
%        \tick Responding information about broker instance
%        \fail Responding topic or partition specific metadat
%        \fail Responding informations about replication 
%    \end{itemize}
%    \fail Offset Management via Request
%    \item Log Persistency
%    \begin{itemize}
%        \tick Persisting messages in log file
%        \tick Guarantee ordering of messages through continuously offsets
%        \tick Keep book of a log file with separate index file 
%        \fail Log Compaction 
%    \end{itemize}
%    \item Optimizations 
%    \begin{itemize}
%        \tick Network throughput 
%        \tick Caching
%    \end{itemize}
%    \fail Message Compression 
%    \fail Consumer Groups 
%    \fail Broker Recovery
%    \fail Broker Replication and Zookeeper Integration
%   \end{itemize}

\section{Broker Performance}
-> Test System

Broker Server Setup
\begin{verbatim}
Intel(R) Xeon(R) CPU E3-1245 v3 @ 3.40GHz
One 7200 RPM SATA drive
16GB of RAM
1Gb Ethernet 
\end{verbatim}

\subsection{Network Throughput}
The socket based communication must not be a bottleneck in a high performance
broker. A producer should be able to send with near the maximal throughput of
its local network card, whereas the broker needs to handle the incoming data
streams as fast as possible. If the limits of a single network connection is
reached, the next step to improve performance is to balance the data
stream on multiple replicated brokers.

To test the network throughput for this thesis, a benchmark producer
which identify the limits is provided.

\begin{lstlisting}
import qualified System.Entropy as E

main = do 
    -- Socket setup 
    -- ....

    randBytes <- E.getEntropy 1000
    let
\end{lstlisting}


The variance seems to be due to Linux's I/O management facilities that batch
data and then flush it periodically.
-> Durchgeführte Tests 

-> Resultat Network Throuput 

-> Resultat Writing to the log 
-> Vergleich Kafka (ref to Blog) 

\begin{table}[h]
\begin{tabular}{|c|c|c|c|}
\hline
{\bf Request Size (with ReqHeader)} & {\bf Message Size {[}Byte{]}} & {\bf Batch Size} & {\bf \begin{tabular}[c]{@{}c@{}}Avg. Throughput\\ {[}MB/s{]}\end{tabular}} \\ \hline
150 B                               & 100                           & 1                & 27                                                                         \\ \hline
1050 B                              & 100                           & 10               & 22                                                                         \\ \hline
10.050 KB                           & 100                           & 100              & 30                                                                         \\ \hline
100.050 KB                          & 100                           & 1000             & 32                                                                         \\ \hline
1 MB                                & 100                           & 10000            & 40                                                                         \\ \hline
550 B                               & 500                           & 1                & 66                                                                         \\ \hline
1050 B                              & 500                           & 2                & 81                                                                         \\ \hline
10.050 B                            & 500                           & 20               & 80                                                                         \\ \hline
100.050 KB                          & 500                           & 200              & 96                                                                         \\ \hline
1 MB                                & 500                           & 2000             & 68                                                                         \\ \hline
1050 B                              & 1000                          & 1                & 104                                                                        \\ \hline
10.050 B                            & 1000                          & 10               & 110                                                                        \\ \hline
100.050 KB                          & 1000                          & 100              & 112                                                                        \\ \hline
1 MB                                & 1000                          & 1000             & 87                                                                         \\ \hline
10.050 B                            & 10000                         & 1                & 117                                                                        \\ \hline
100.050 KB                          & 10000                         & 10               & *                                                                          \\ \hline
1 MB                                & 10000                         & 100              & *                                                                          \\ \hline
\end{tabular}
\end{table}

