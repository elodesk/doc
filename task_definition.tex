\chapter{Task Description}
\section*{Betreuer}
\begin{itemize}
    \item J.M.Joller, Dozent für Informatik
\end{itemize}

\section*{Ausgangslage, Problembeschreibung}
Funktionale Programmieransätze werden zunehmend auch in klassischen
Programmiersprachen eingebaut:  Lambdas, Streaming API um nur zwei Beispiele zu
nennen. Sprachen wie Scala mit ihrem sehr reichen Typensystem sind aber in der
Regel nur nicht und in ganzer Tiefe verständlich wenn der Entwickler über
vertiefte Kenntnisse einer strikt funktionalen Programmiersprache besitzt. Die
Erfahrungen in Industrieprojekten mit Scala zeigen, dass nur sehr gut
ausgebildete Entwickler brauchbaren und wartbaren Scala Code schreiben können. 

Auf Grund dieser Beobachtungen basiert der MAS Kurs über funktionale
Programmierung auf Haskell, genau wie der Multi-Paradigmen Teilmodul im Modul
Compilerbau.

\section*{Aufgabenstellung}
Ein Ziel der Arbeit ist zunächst die Erarbeitung einer Übersicht zum
Entwicklungsstand der Message basierten Middleware (MON), speziell der Event
Streaming Ansätze. 

\par
\begingroup
\leftskip4em
\rightskip\leftskip
Wesentliches Ergebnis dieses Teil ist die Einordnung des in Scala geschrieben
Message Brokers "Kafka" von Apache / Linked. Kafka hat die Ansprüche "fast,
scalable, durable, distributed by design" zu sein (Dokumentation von Apache
Kafka: \url{http://kafka.apache.org/}).
\par
\endgroup

Ein weiteres  Ziel ist die Definition und Implementierung des
Kommunikations-Protokolls ("Wire-Protocol") von Kafka mithilfe von Haskell.

\par
\begingroup
\leftskip4em
\rightskip\leftskip
Dieser Teil der Arbeit ist zentral, da über das Protokoll unterschiedlich
(Programmier-Sprachen, Betriebssystem) entwickelte Clients mit Kafka
kommunizieren können.
\par
\endgroup

Das dritte Ziel ist die Implementierung von Grundfunktionen von Kafka in
Haskell.

\par
\begingroup
\leftskip4em
\rightskip\leftskip
Da Kafka vom Design her auf starke Dezentralisierung ausgerichtet ist, verwendet
Kafka "ZooKeeper" (Apache) auf den unteren Layern. Diese Funktionalität
(Multi-Cluster) übersteigt die Möglichkeiten einer Bachelorarbeit (BA). Die BA
beschränkt sich auf einen Cluster.
\par
\endgroup

Ein wichtiges Ziel ist der Performance-Vergleich der Haskell mit der Scala
Implementierung.

\par
\begingroup
\leftskip4em
\rightskip\leftskip
Dazu muss Kafka installiert und ein sinnvoller Benchmark definiert werden.
\par
\endgroup

\section*{Zur Durchführung}
Die Bearbeitung der Aufgabe setzt eine Einarbeitung in technischen Grundlagen
von Message Brokern und in die Programmierung mit Haskell  voraus.

Solide Programmierkenntnisse und die Bereitschaft, sich rasch neues
Domänenwissen anzueignen, werden vorausgesetzt. 

Wegen der variablen Breite der Aufgabe eignet sich das Thema gut für eine
Bearbeitung durch zwei Studierende. 

Mit dem Betreuer finden in der Regel wöchentliche Besprechungen statt.
Zusätzliche Besprechungen sind nach Bedarf durch die Studierenden zu
veranlassen. Alle Besprechungen sind von den Studenten mit einer Traktandenliste
vorzubereiten und die Ergebnisse sind in einem Protokoll zu dokumentieren, das
dem Betreuer per E-Mail zugestellt wird.

Für die Durchführung der Arbeit ist ein Projektplan zu erstellen. Dabei ist auf
einen kontinuierlichen und sichtbaren Arbeitsfortschritt zu achten. An
Meilensteinen gemäss Projektplan sind einzelne Arbeitsresultate in vorläufigen
Versionen abzugeben. Über die abgegebenen Arbeitsresultate erhalten die
Studierenden ein vorläufiges Feedback. Eine definitive Beurteilung erfolgt auf
Grund der am Abgabetermin abgelieferten Implementierung und Dokumentation. 

\section*{Dokumentation und Abgabe}
Wegen der beabsichtigten Verwendung der Ergebnisse in der Lehre wird auf
Vollständigkeit sowie (sprachliche und grafische) Qualität der Dokumentation
erhöhter Wert gelegt.

Die Dokumentation zur Projektplanung und –Verfolgung ist gemäss den Richtlinien
der Abteilung Informatik anzufertigen. Die Detailanforderungen an die
Dokumentation der Recherche- und Entwicklungsergebnisse werden entsprechend dem
konkreten Arbeitsplan festgelegt.

Die Dokumentation ist vollständig auf CD in drei Exemplaren abzugeben.
Neben der Dokumentation sind abzugeben:

\begin{itemize}
    \item ein Poster zur Präsentation der Arbeit
    \item alle zum Nachvollziehen der Arbeit notwendigen Ergebnisse und Daten (Quellcode,
        Buildskripte, Testcode, Testdaten usw.)
    \item Material für eine Abschlusspräsentation (ca. 20’)
\end{itemize}

\section*{Termine}
\begin{table}[H]
\begin{tabular}{|l|p{12cm}|}
\cline{1-2}
{\bf FS 2015}     & {\bf Beginn der Bachelorarbeit, Ausgabe der Aufgabenstellung
durch die Betreuer}
  \\ \cline{1-2}
1. Semesterwoche  & Kick-off Meeting
  \\ \cline{1-2}
2. Semesterwoche  & Abgabe der Projektplanung (Entwurf), einschliesslich ggf. zu
beschaffender Hardware
  \\ \cline{1-2}
4. Semesterwoche  & \begin{tabular}[c]{@{}l@{}}Abgabe eines vorläufigen
Technologierechercheberichts und eines\\ detaillierten Vorschlags für einen
Arbeitsplan. Festlegung des\\ Projektplans mit dem Betreuer\end{tabular}   \\
\cline{1-2}
6. Semesterwoche  & \begin{tabular}[c]{@{}l@{}}Abgabe eines Umsetzungsvorschlags
für die Experimentierumgebung\\ (Funktionsumfang, Aufwandsabschätzung)\\
Abstimmung und Festlegung mit dem Betreuer\end{tabular}              \\
\cline{1-2}
7. Semesterwoche  & Abgabe Anforderungs- und Domainanalyse für die
Experimentierumgebung
  \\ \cline{1-2}
10. Semesterwoche & Reviewmeeting Softwaredesign für die Experimentierumgebung
  \\ \cline{1-2}
13. Semesterwoche & Vorstellung der Implementierung (Arbeitsstand)
  \\ \cline{1-2}
Anfang Juni 2015  & \begin{tabular}[c]{@{}l@{}}Abgabe der Kurzbeschreibung für
die Diplomarbeitsbroschüre und des\\ A0-Posters zur Kontrolle an den
Betreuer\end{tabular}                                                     \\
\cline{1-2}
Mitte Juni 2015   & Abgabe des Berichtes an den Betreuer (bis 12:00 Uhr),Abgabe
des Posters im Studiengangsekretariat
  \\ \cline{1-2}
\end{tabular}
\end{table}
\captionof{table}{Termine}

Die detaillierten Termine werden durch die Abteilung für Informatik festgelegt.
Sie erhalten zur gegebenen Zeit ein email vom  Abteilungs-Sekretariat.

\section*{Literaturhinweise}
\begin{itemize}
    \item [1] Apache Kafka: \url{http://kafka.apache.org}
    \item [2] Apache ZooKeeper: \url{http://zookeeper.apache.org}
    \item [3] Haskell: \\
           Hutton Programming Haskell
           \url{http://www.cs.nott.ac.uk/~gmh/book.html}\\
            Miran Lipovača; Learn you a Haskell
            \url{http://leanyouahaskell.com} \\
            Bryan O'Sullivan, Don Stewart, and John Goerzen  Real World Haskell 
            \url{http://book.realworldhaskell.org}
\end{itemize}

