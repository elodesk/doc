\chapter{Management Summary}

\section{Introduction}

This thesis aims to adapt the concept of Apache Kafka and build a messaging
system, namely Haskell Message Broker (HMB), in the functional programming
language Haskell. Thereby the focus lies on the implementation of a stable
(related to networking) server application system as well as on the building a
log subsystem, which in Apache Kafka is considered to be the most important
feature. As Kafka comes with its own wire-protocol a part of this thesis will
focus on a fully compatible implementation of the Apache Kafka Protocol in
Haskell, including a client library allowing Haskell applications to take use of
the protocol implementation and communicate with Apache Kafka or HMB.

\section{Approach}

Becoming familiar with the state of the art in messaging, especially event
streaming, was inevitably necessary to eventually create a messaging system by
ourselves. As essential part of this prestudy we analysed the approach and
functionality of Apache Kafka. In this first third of the thesis we also learned
the functional paradigm together with the programming language Haskell,
intensively. After the prestudy phase, an architecture prototype to demonstrate
some very basic functionality of a message broker was elaborated. With this in
our hands, we then continued by working out the details for the protocol
implementation and the server application. A code review byexpert Simon Meier
helped us to tweak our code and improve efficiency. Finally, we tested our
system under high load to optimize performance of our application.

\section{Results}

The first result of this thesis is the prestudy documentation, which resumes
gathered knowledge in the familiarization phase of this work. It gives an
insight in messaging fundamentals and  takes a closer look to Apache Kafka and
related topics. It can be offered for using as academic amendment for existing
lectures. Another result is the implementation of the Kafka protocol in Haskell.
The design decision of separating protocol related code from the broker
implementation leads to a isolated product which can be used as library for
different projects. The open sourced code has already been praised by the
Haskell community and found its contributors who helped uncovering minor issues.
Finally, the resulting broker application provides a server with basic
functionality in networking and persisting messages. It adapts some features of
Apache Kafka and provides the ability to produce and consume data. It supports
Kafka clients as it is based on the protocol implementation mentioned above.
Simple console clients are provided to demonstrate the functionality.

\section{Outlook}

- broker results promissing
- very extendable/scalable implementation base and architecture
- with further work one could build the current prototype to an extraordinary
broker system.

As for now, the highlight remains the protocol implementation which has already
been praised by the Haskell community and found its contributors they helped
uncovering minor issues.
