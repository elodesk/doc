\chapter{Introduction to Distributed Log}
\label{intro-log}
\section{What is a Distributed Log?}
- Very Simple data structure 
- append-only, totally-ordered sequence of records ordered by time (entries on
the left are older then entries on the right)
- Each entry is assigned a unique sequential log entry number. Number can be
thought as "timestamp" -> decoupling from any time system) 
- record what happend and when 
- nothing to do with "application logging", as traces an application writes
out. Application Logs are designed for human read wheras log for distributed
systems are built for programmatic accessa
- producing a persistent, re-playable record of history
\\
Origin:
- origing from databases where logs used to replicating data between databases.
Database Logs include information about records of what happend. 
- Logs as mechanism for data subscription is ideal for supporting all kings of
messaging, data flow and real-time data processing. 

Main Problem: managing distributed, concurrent changes in state.(e.g. Version
Control). Log works like version control: sequence of patches = log. When you
update you pull down just the patches an apply them to your current snapshot 

Deterministic Approach for distributed systems: 
- if you feed two deterministic pieces of code the same
input log, they will produce the same output. Applying the same log on a
replicated processes, the processes will remaining consistent across replicas. 
- Reduce the problem of making multiple machines all do the same thing to the
problem of implementing a distributed consistent log to feed these processes
input.
- State of a replica can be easy determined: describe each replica by a single
number, the timestamp for the maximum log entry it has processed. This timestamp
combined with the log uniquely captures the entire state of the replica. 

PAXOS: 
- log can be models the Problem of data consensus.  
- Paxos family of algorithm 

<`0`>
<`0`>
