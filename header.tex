\documentclass[11pt,oneside,a4paper,parskip,listof=numbered]{scrreprt}
\usepackage[inner=2.5cm,outer=2cm,top=2cm,bottom=2cm,includefoot]{geometry}

\usepackage{cmbright}
\setlength{\parindent}{0pt}
\setlength{\parskip}{6pt plus 2pt minus 1pt}
\setlength{\emergencystretch}{3em}  % prevent overfull lines

\usepackage[T1]{fontenc}
\usepackage[utf8]{inputenc}
%/usepackage{ngerman}
\usepackage{graphicx}
\usepackage{color}
\usepackage{ccicons}
\usepackage{rotating}
\usepackage[colorlinks=true, linkcolor=blue, urlcolor=blue, citecolor=blue]{hyperref}
\usepackage{pdfpages}
\usepackage{todonotes}
\usepackage{tabularx}
\usepackage{xcolor,colortbl}
\usepackage{pdflscape}
\usepackage{soul}
\usepackage{glossaries}
\usepackage{amsmath}
\usepackage{amssymb}
\usepackage{pifont}
\usepackage{fourier}
\usepackage{MnSymbol}
\usepackage{float} % Bilder fix positionieren
\usepackage{appendix}
\usepackage{booktabs}
\usepackage{makeidx}
\makeindex
\usepackage{fancybox}
\usepackage[nottoc,numbib]{tocbibind}
\usepackage{enumitem}
% \usepackage{slashbox} %halbierte Tabellenzellen
% \setlist{nolistsep}
\setlist[itemize]{noitemsep, topsep=0pt}

\usepackage{tikz}
\usepackage{pgf-umlsd}
\usepgflibrary{arrows} % for pgf-umlsd

\usepackage{hyperref}
\newcommand\fnurl[2]{%
  \href{#2}{#1}\footnote{\url{#2}}%
}

\usepackage{bbding}
\newcommand*\tick{\item[\Checkmark]}
\newcommand*\fail{\item[\XSolidBrush]}

% \let\stdsection\section
% \renewcommand\section{\newpage\stdsection}

\subject{\SUBTITLE}
\title{\TITLE}
\author{\AUTHOR \\ \EMAIL}

\hypersetup{
  pdftitle    = {\TITLE},
  pdfsubject  = {\SUBTITLE},
  pdfauthor   = {\AUTHOR, \EMAIL},
  pdfkeywords = {\KEYWORDS} ,
  pdfcreator  = {pdflatex},
  pdfproducer = {LaTeX with hyperref}
}


\newcommand{\logoLX}{\fcolorbox{white}{red}{\textcolor{white}{LX}}}
\newcommand{\logoSH}{\fcolorbox{white}{maroon}{\textcolor{white}{SH}}}
\newcommand{\logoSE}{\fcolorbox{white}{orange}{\textcolor{white}{SE}}}
\newcommand{\logoXJ}{\fcolorbox{white}{darkblue}{\textcolor{white}{XJ}}}
\newcommand{\logoXC}{\fcolorbox{white}{darkgreen}{\textcolor{white}{XC}}}
\newcommand{\logoMF}{\fcolorbox{white}{lila}{\textcolor{white}{M4}}}
\newcommand{\logoMS}{\fcolorbox{white}{violett}{\textcolor{white}{M6}}}
\newcommand{\logoMN}{\fcolorbox{white}{pink}{\textcolor{white}{MN}}}

\newcommand{\verwundbar}{\hfill \fcolorbox{black}{red}{(Verwundbar \bomb)}}
\newcommand{\nichtverwundbar}{\hfill \fcolorbox{black}{green}{(Nicht verwundbar~\sun)}}
\newcommand{\zelleverwundbar}{\cellcolor{red} \bomb}
\newcommand{\zellenichtverwundbar}{\cellcolor{green} \sun}
\newcommand{\zelleteilweiseverwundbar}{\cellcolor{yellow} \danger }
\newcommand{\gegenmassnahme}{$\rightarrow$ Die Gegenmassnahme ist in diesem Kapitel zu finden: }
\newcommand{\keinegegenmassnahme}{$\rightarrow$ Diese Verwundbarkeit lässt sich mit keiner Gegenmassnahme beheben.}

\definecolor{yellow}{RGB}{240,173,78}
\definecolor{orange}{RGB}{255,128,0}
\definecolor{lila}{RGB}{128,128,192}
\definecolor{violett}{RGB}{64,0,128}
\definecolor{pink}{RGB}{255,128,192}

% Code-Listing
\definecolor{gray}{rgb}{0.4,0.4,0.4}
\definecolor{darkblue}{rgb}{0.0,0.0,0.6}
\definecolor{cyan}{rgb}{0.0,0.6,0.6}
\definecolor{green}{RGB}{92,184,92}
\definecolor{red}{RGB}{217,83,79}
%\definecolor{light-gray}{gray}{0.95}
%\definecolor{lbcolor}{rgb}{0.9,0.9,0.9}
\usepackage{listings}
\definecolor{maroon}{rgb}{0.5,0,0}
\definecolor{darkgreen}{rgb}{0,0.5,0}
\definecolor{sh_comment}{rgb}{0.12, 0.38, 0.18 } %adjusted, in Eclipse: {0.25, 0.42, 0.30 } = #3F6A4D
\definecolor{sh_keyword}{rgb}{0.37, 0.08, 0.25}  % #5F1441
\definecolor{sh_string}{rgb}{0.06, 0.10, 0.98} % #101AF9

% \lstset{numbers=left,
%     language=Java,                               % oder C++, Pascal, {[77]Fortran}, ...
%     basicstyle=\ttfamily,                        % Textgröße des Standardtexts
%     keywordstyle=\ttfamily\color{red},           % Formattierung Schlüsselwörter
%     commentstyle=\ttfamily\color{green},         % Formattierung Kommentar
%     stringstyle=\ttfamily\color{blue},           % Formattierung Strings
%     breaklines=true,                             % Umbruch langer Zeilen
%     showstringspaces=false,                      % Spezielles Zeichen für Leerzeichen
% }
%\lstset {
% frame=rlbt,
% rulesepcolor=\color{black},
% showspaces=false,showtabs=false,tabsize=2,
% %numberstyle=\tiny,numbers=left,
% basicstyle=\ttfamily\footnotesize,
% stringstyle=\color{sh_string},
% keywordstyle = \color{sh_keyword}\bfseries,
% commentstyle=\color{darkgreen}\itshape,
% captionpos=b,
% %lineskip=-0.1em,
% showstringspaces=false,
% escapebegin={\lstsmallmath}, escapeend={\lstsmallmathend},
% breaklines=true, % breakatwhitespace=true,
% prebreak=\raisebox{0ex}[0ex][0ex]{\ensuremath{\rhookswarrow}},
% % postbreak=\raisebox{0ex}[0ex][0ex]{\ensuremath{\rcurvearrowse\space}}
%}
%\lstnewenvironment{code}{\lstset{language=Haskell}}{}
%
%\newcommand{\inputlisting}[2][]{%
%  \lstinputlisting[caption={\texttt{\detokenize{#2}}},#1]{#2}%
%}

%\lstdefinelanguage{XML}
%{
%  basicstyle=\ttfamily\footnotesize,
%  morestring=[b]",
%  moredelim=[s][\bfseries\color{maroon}]{<}{\ },
%  moredelim=[s][\bfseries\color{maroon}]{</}{>},
%  moredelim=[l][\bfseries\color{maroon}]{/>},
%  moredelim=[l][\bfseries\color{maroon}]{>},
%  morecomment=[s]{<?}{?>},
%  morecomment=[s]{<!--}{-->},
%  commentstyle=\color{darkgreen},
%  stringstyle=\color{blue},
%  identifierstyle=\color{red}
%}
\lstset{
  frame=none,
  xleftmargin=25pt,
  stepnumber=1,
  numbers=left,
  numbersep=5pt,
  numberstyle=\ttfamily\tiny\color[gray]{0.3},
  belowcaptionskip=\bigskipamount,
  captionpos=b,
  escapeinside={*'}{'*},
  language=haskell,
  tabsize=2,
  emphstyle={\bf},
  commentstyle=\it,
  stringstyle=\mdseries\rmfamily,
  showspaces=false,
  keywordstyle=\bfseries\rmfamily,
  columns=flexible,
  basicstyle=\small\sffamily,
  showstringspaces=false,
  morecomment=[l]\%,
  mathescape=false
}

\lstloadlanguages{Haskell}
\lstnewenvironment{code}
    {\lstset{}%
      \csname lst@SetFirstLabel\endcsname}
    {\csname lst@SaveFirstLabel\endcsname}
    \lstset{
      basicstyle=\small\ttfamily,
      flexiblecolumns=false,
      basewidth={0.5em,0.45em},
      literate={+}{{$+$}}1 {/}{{$/$}}1 {*}{{$*$}}1 {=}{{$=$}}1
               {>}{{$>$}}1 {<}{{$<$}}1 {\\}{{$\lambda$}}1
               {\\\\}{{\char`\\\char`\\}}1
               {->}{{$\rightarrow$}}2 {>=}{{$\geq$}}2 {<-}{{$\leftarrow$}}2
               {<=}{{$\leq$}}2 {=>}{{$\Rightarrow$}}2 
               {\ .}{{$\circ$}}2 {\ .\ }{{$\circ$}}2
               {>>}{{>>}}2 {>>=}{{>>=}}2
               {|}{{$\mid$}}1               
    }

% Alternative checkbox list
% \usepackage{wasysym}
% 
% \setlength{\marginparwidth}{1.2in}
% \let\oldmarginpar\marginpar
% \renewcommand\marginpar[1]{\-\oldmarginpar[\raggedleft #1]%
% {\raggedright #1}}
% 
% \newenvironment{checklist}{%
%   \begin{list}{}{}% whatever you want the list to be
%   \let\olditem\item
%   \renewcommand\item{\olditem -- \marginpar{$\Box$} }
%   \newcommand\checkeditem{\olditem -- \marginpar{$\CheckedBox$} }
% }{%
%   \end{list}
% }


\makeglossaries

\chapter{Glossar}

\begin{description}
  \item[Artifact] Gegenstand
  \item[Socket] Communication endpoint to which an application can write data
that are to be sent out over the underlying network, and from which incoming end
data can be read. \cite{TAN06}
\end{description} 

